\documentclass[11pt]{amsart}
 
\setlength{\topmargin}{-0.3in} % usually -0.25in
\addtolength{\textheight}{.9in} % usually 1.25in
\addtolength{\oddsidemargin}{-0.45in}
\addtolength{\evensidemargin}{-0.45in}
\addtolength{\textwidth}{1.0in} %\setlength{\parindent}{0pt}

\newcommand{\normalspacing}{\renewcommand{\baselinestretch}{1.1}\tiny\normalsize}
\normalspacing

% macros
\usepackage{amssymb,xspace,alltt,verbatim,pgfplots}
%\usepackage[final]{graphicx}
\usepackage{fancyvrb}

\newtheorem*{lem*}{Lemma}
\newtheorem*{thm}{Theorem}

\newcommand{\bs}{\mathbf{s}}
\newcommand{\bu}{\mathbf{u}}
\newcommand{\bv}{\mathbf{v}}
\newcommand{\bx}{\mathbf{x}}
\newcommand{\bbf}{\mathbf{f}}

\newcommand{\CC}{{\mathbb{C}}}
\newcommand{\RR}{{\mathbb{R}}}
\newcommand{\eps}{\epsilon}
\newcommand{\ZZ}{{\mathbb{Z}}}
\newcommand{\ZZn}{{\mathbb{Z}}_n}
\newcommand{\NN}{{\mathbb{N}}}
\newcommand{\ip}[2]{\mathrm{\left<#1,#2\right>}}
\newcommand{\spn}{\operatorname{span}}

\newcommand{\emach}{\epsilon_{\mathrm{mach}}}

\newcommand{\prob}[1]{\bigskip\noindent\large\textbf{#1.} \, \normalsize}
\newcommand{\apart}[1]{\textbf{(#1)} \quad}
\newcommand{\epart}[1]{\medskip\noindent \textbf{(#1)} \quad}


\begin{document}
\thispagestyle{empty}
\Large \noindent \underline{\textbf{SAMPLE}} \hfill\underline{\textbf{SAMPLE}}

\scriptsize \noindent February 2025  \hfill  \tiny [AUTHOR: Bueler]
\normalsize\bigskip

\centerline{\large DRAFT \textbf{Optimization Comprehensive Exam}}
\bigskip

\centerline{\textsc{Do \textbf{FIVE (5)} of the following 7 problems, and clearly indicate which ones.}}
\smallskip

\thispagestyle{empty}

\prob{A}  \apart{a} FIXME SIMPLIFY REGION Consider the linear programming problem
\begin{alignat*}{2}
    \text{minimize}   && \qquad z = - 2 &x_1 - 3 x_2 \\
    \text{subject to} && -x_1 + x_2 &\ge -3 \\
                      && - 2 x_1 + x_2 &\le 1 \\
                      && x_1 - 2 x_2 &\ge -5 \\
                      && x_1 + x_2 & \le 7 \\
                      && x &\ge 0
\end{alignat*}

Sketch the feasible set for the above problem.  Be sure to label the axes, label the feasible set $S$, and give coordinates for each extreme point of $S$.

\epart{b} Put the above problem in standard form.

\epart{c} If you start at the extreme point FIXME, and apply the simplex method, which will be the next extreme point which will be visited?


\prob{B}  FIXME ARGUE THAT IF THERE ARE GENERAL LINEAR CONSTRAINTS $a_i^T x = b_i$ and $a_i^T \ge b_i$ THEN $p$ IS A FEASIBLE DIRECTION IFF \dots AND (b) ONLY CAN GO SO FAR $x = x + \alpha p$


\prob{C}  Consider the problem
\begin{alignat*}{2}
    \text{minimize}   &&  f(x) &= (x_1-1)^2 + (x_2+1)^2 \\
    \text{subject to} && \quad x_1^2 + x_2^2 &\le 4 \\
                      &&        x_2 &\ge 0
\end{alignat*}

\epart{a}  Sketch the feasible set and explain informally, perhaps using contours of $f$, why $x_*=(1,0)^\top$ is the solution.

\epart{b}{  Write the constraints in the form $g_i(x)\ge 0$.  Compute the Lagrangian and its gradient.  For each of the points $A = (0,0)^\top$, $B=(0,2)^\top$, and $C=(1,0)^\top$ compute the values of the Lagrange multipliers $\lambda_i$ satisfying the zero-gradient condition.  Address whether these points satisfy the first-order optimality conditions, that is, whether they are candidates for a local minimizer.  Show in particular that $C$ satisfies all the first-order conditions.


\prob{D}  Consider the 2D unconstrained minimization problem
    $$\text{minimize} \quad f(x)=3\sin(x_1)+\cos(x_2) + \frac{1}{20} \left(x_1^2 - x_1 x_2 + 2 x_2^2\right).$$

\epart{a} Compute the gradient and the Hessian.

\epart{b} The surface $z=f(x)$ has many local maxima and minima.  Choose an algorithm which would quickly find a local minima, and write a pseudocode for this algorithm.  Consider algorithms which use the Hessian and which have guarantees of finding critical points.

\epart{c} Describe an algorithm which would find all of the local minima in a closed rectangle, for example $R=\{-10<x_1\le 10, -10<x_2\le 10\}$.


\prob{E}  Consider nonlinear optimization problems on $x\in \RR^n$ which have standard-form linear constraints:
    $$\begin{matrix}
    \text{minimize} \qquad & f(x) \\
    \text{subject to} \qquad & Ax = b \\
                      & x \ge 0
    \end{matrix}$$
Assume $f$ is smooth and that there are $m$ scalar constraint equations, i.e.~$b\in \RR^m$ and $A\in \RR^{m\times n}$.

In 2D ($n=2$) there are three possibilities for the dimension of the feasible set.  The cartoons below illustrate these three possibilities in the cases where the feasible sets $S$ are non-empty, generic, and bounded.  (``Bounded'' applies for $m>0$.)

\bigskip
\begin{tikzpicture}[scale=0.65]
\filldraw [gray!50] (0,0) -- (4.9,0) -- (4.9,3.9) -- (0,3.9) -- cycle;
\draw [->] (-0.5,0)--(5,0) node[right] {$x_1$};
\draw [->] (0,-0.5)--(0,4) node[above] {$x_2$};
\node at (2,2) {\Large $S$};
\node at (2,-1) {$m=0$};
\end{tikzpicture}
\qquad
\begin{tikzpicture}[scale=0.65]
\draw [->] (-0.5,0)--(5,0) node[right] {$x_1$};
\draw [->] (0,-0.5)--(0,4) node[above] {$x_2$};
\draw [very thick] (0,3) -- (3.5,0);
\draw [->] (2.1,2.1) node[xshift=2mm,yshift=2mm] {\Large $S$} -- (1.7,1.7);
\node at (2,-1) {$m=1$};
\end{tikzpicture}
\qquad
\begin{tikzpicture}[scale=0.65]
\draw [->] (-0.5,0)--(5,0) node[right] {$x_1$};
\draw [->] (0,-0.5)--(0,4) node[above] {$x_2$};
\draw [black,fill=black] (1.5,1.5) circle (0.5mm);
\draw [->] (2.1,2.1) node[xshift=2mm,yshift=2mm] {\Large $S$} -- (1.7,1.7);
\node at (2,-1) {$m=2$};
\end{tikzpicture}

\bigskip
For 3D ($n=3$) there are four possibilities $m=0,1,2,3$.  Sketch the four corresponding cartoons.  These cartoons should have the same annotations as the 2D versions above.


\prob{F}  Suppose $c\in\RR^n$ is a nonzero vector and consider the problem
\begin{alignat*}{2}
    \text{minimize}   &&  z = c^\top &x \\
    \text{subject to} && \quad \sum_{i=1}^n x_i^2 &= 1
\end{alignat*}
where $x\in\RR^n$.  Note there is a single equality constraint, which can be written $\|x\|^2=1$.

\epart{a}  By arguing informally explain why the solution is
    $$x_* = - \frac{c}{\|c\|}.$$
Use a sketch of the $n=2$ case to explain.

\epart{b}  Compute the Lagrangian and state the first-order necessary conditions for the above problem.

\epart{c}  Solve the conditions in \textbf{(b)} algebraically to confirm the solution in part \textbf{(a)}.  How many points $(x_*,\lambda_*)$ are there which satisfy the first-order necessary conditions?


\prob{G}  (\emph{This problem asks you to write the general KKT conditions.})  Consider the general nonlinear constrained optimization problem
\begin{alignat*}{2}
    \text{minimize}   &&  &f(x) \\
    \text{subject to} && \qquad g_i(x) &= 0, \quad i=1,\dots,\ell \\
                      &&       h_i(x) &\ge 0, \quad i=1,\dots,m
\end{alignat*}

\epart{a}  State the Lagrangian for this problem.

\epart{b}  Suppose $x_*$ is a local minimizer which is also a regular point.  State the first-order necessary conditions for the above problem.



\end{document}
