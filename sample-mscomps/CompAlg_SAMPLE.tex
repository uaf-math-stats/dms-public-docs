% !TEX TS-program = pdflatexmk
\documentclass[12pt]{amsart}
\usepackage[margin=.8in]{geometry}
\usepackage{amsmath,amssymb,amsthm, latexsym, mathrsfs, pdfsync, multicol,
fancybox, fancyhdr,
graphicx, enumerate,
subfig, tikz}

\newcommand{\blankbox}[2]{\fbox{\rule{#1}{0in}\rule{0in}{#2}}}
%		Problem and Part
\newcounter{problemnumber}
\newcounter{partnumber}
\newcommand{\Problem}{\stepcounter{problemnumber}\setcounter{partnumber}{0}
       \item[\fbox{\parbox{2em}{\hfil\theproblemnumber\hfil}}]}
\newcommand{\Part}{\stepcounter{partnumber}\item[(\alph{partnumber})]}
\newcommand{\Points}[1]{(#1 points) \quad}

\newcommand{\testname}{Final Examination}
\newcommand{\class}{Math F200X}
\newcommand{\quarter}{Spring 2011}

\newcommand{\RR}{\ensuremath{\mathbb R} }
\newcommand{\NN}{\ensuremath{\mathbb N} }
\newcommand{\ZZ}{\ensuremath{\mathbb Z} }
\newcommand{\QQ}{\ensuremath{\mathbb Q} }
\newcommand{\CC}{\ensuremath{\mathbb C} }
\newcommand{\normale}{\trianglelefteq}
\newcommand{\normal}{\triangleleft}
\newcommand{\bc}{\begin{center}}
\newcommand{\ec}{\end{center}}
\newcommand{\be}{\begin{enumerate}}
\newcommand{\ee}{\end{enumerate}}
\newcommand{\bi}{\begin{itemize}}
\newcommand{\ei}{\end{itemize}}
\newcommand{\bs}{\begin{slide}}
\newcommand{\es}{\end{slide}}
\newcommand{\bx}{\begin{exercise}}
\newcommand{\ex}{\end{exercise}}
\newcommand{\ol}[1]{\overline{#1}}
\newcommand{\oimp}[1]{\overset{#1}{\Longleftrightarrow}}
\newcommand{\bv}[1]{\ensuremath{ \mathbf{\vec{#1}}} }
%\newcommand{\eP}{\varphi}

\newcommand\dsp{\displaystyle}

\newcommand\eP{\varphi}
\newcommand\ns{\trianglelefteq} 
\newcommand\ldiv{\, \bigg | \,}
\newcommand{\Mod[1]}{~( \operatorname{mod} \,#1)}
\newcommand{\Aut}{\operatorname{Aut}}
\newcommand\FF{{\mathbb F}}

\newcommand{\tcm}{\textcolor{magenta}}


\usepackage{textcomp}

\newcommand{\tc}{\text{:}} 


\lhead{\sc Math 631F}
\chead{\Large \sc Sample MS Comprehensive} 
\rhead{\sc Spring 2025}
\cfoot{}
\pagestyle{fancy}
%
\begin{document}
\thispagestyle{fancy}


The actual test will consist of a subset of these problems. You will have two hours to complete the test.

{\bf Instructions:}

\begin{itemize}

\item 
This test will be closed note and closed book.
\item
In order to receive full credit, you must {\bf show your work.}  
\item
Raise your hand if you have a question.
\item Unless told explicitly otherwise, {\bf in each ring there is a multiplicative identity $1\ne 0$.}
\end{itemize}
On this sample exam, there are many problems which serve as a model for a problem, but where the details may change for the actual comprehensive exam (e.g., polynomials or integers that appear in problem statements may change).
%
\bc \Large\textsc{Part 1: Short Answer} \ec

Complete each of the exercises in this section. (On the actual comprehensive exam,  you'll have  approximately five problems to complete.)
\begin{enumerate}
\Problem Consider the cylic group $C_{4900} = \langle x \rangle$ of order $4900 = 2^2 \cdot 5^2 \cdot 7^2$. 

\Part Give the number of generators of $C_{4900}$.

\Part List explicitly the elements $x^a$, with $0 \le a \le 4899$, of order $10$.  

\tcm{\emph{Answer:}}  $\vert x^a \vert = 10$ if $a = \underline{\hskip 12cm }$.

\medskip

(If it helps, you can simply give the prime factorizations of $a$.  I am not interested in your ability to multiply integers.)



\Problem Consider the cyclic groups $\ZZ/30\ZZ$ and $C_{18} = \langle x \rangle$ of orders $30$ and $18$ respectively, and
suppose that 
\begin{align*}
\varphi_a : \ZZ/30\ZZ &\to C_{18}\\
 1  &\mapsto x^a
\end{align*}
extends to a well-defined group homomorphism from $\ZZ/30\ZZ$ to $C_{18}$.

\Part List the values of $a$ with $0 \le a \le 17$ for which this is true.  (I.e. The map defines a well-defined group
homomorphism.)

\Part Give a brief explanation why such a well-defined group homomorphism can not be surjective.


\Problem Consider the symmetric group $G = S_{7}$ and let $\sigma = (1 \ 2 \ 3 \ 6 \ 5 \ 4 \  7 )$ be a $7$-cycle.  

\Part Express $\sigma$ as the product of (not necessarily disjoint) transpositions.

\Part Compute the number of conjugates of $\sigma$ in $S_7$.

\Part Let $\tau$ be the $7$-cycle $(3 \ 7 \ 1 \ 4 \ 5 \ 6 \ 2)$.  Give an element $\alpha$ that conjugates
$\sigma$ to $\tau$, i.e.  give $\alpha$ such that $\alpha \sigma {\alpha}^{-1} = \tau$.

\Part Noting that $S_7$ acts on itself by conjugation, explicitly use the Orbit-Stabilizer theorem to find the 
size of the stabilizer of $\sigma$ under this action and the elements of the Stabilizer subgroup
of $S_7$.  

\tcm{The stabilizer of $\sigma$ in this context  is better known as  \underline{\hskip 3cm}. (Using appropriate notation in place of words 
here is fine.)}

\Part Noting that $\sigma \in A_7$, what is the size of the conjugacy class of $\sigma$ in $A_7$?
Stated otherwise, how many conjugates in $A_7$ does $\sigma$ have?  Briefly, state a result that justifies
your answer.

\tcm{\emph{Answer:} The number of conjugates of $\sigma$ in $A_7$ is \underline{\hskip 2cm}} 

\medskip

\tcm{because ....}



\Problem Suppose that $A$ is an Abelian group of order $200 =  2^3 \cdot 5^2$.  Give the isomorphism
classes for $A$ in a table below.  In the left hand column, give the elementary divisor decomposition
and in the right hand column, give the invariant factor decomposition.  {\bf Groups on the same row should
be isomorphic.}  You do not need to show your work.

\Problem  Give the number of non-isomorphic Abelian groups of order $400 = 2^4 \cdot 5^2$.


\Problem Prove that there are no simple groups of order 56.

\Problem Give the definition of a nilpotent element in a ring $R$.  Then
prove that the set of nilpotent elements in $M_2(\QQ)$ is {\bf not} an ideal.

\Problem  Suppose $G$ is a non-cyclic group of order $205 = 5\cdot 41$.  Give, with proof, the number
of elements of order $5$ in $G$.

\Problem 
Find {\bf ALL} solutions $x$ in the integers to the simultaneous congruences. 
\begin{align*}
x \equiv \ &7 \mod{11}\\
x \equiv \ &2 \mod{5}
\end{align*}

\Problem Draw the lattice diagram of prime ideals for the polynomial ring $\QQ[x]$.  \emph{Note:}  There are infinitely many prime ideals so you  will need a way to indicate them all.
\Problem Suppose $H$ a subgroup of $G$ of index 2. Show that $H\normal G$.
\Problem Suppose $\mathbb F$ is a field. Prove that $\mathbb F[x]$ is a principal ideal domain.
\Problem List all abelian groups (up to isomorphism) of order 72.
\Problem Let $G$ be a group. 
\Part Let $G$ be a group, $Z(G)$ the center of $G$. Prove that if $G/Z(G)$ is cyclic, then $G$ is abelian.
\Part Suppose $G$ is a group of order $p^2$, where $p$ is a prime. Prove that $G$ is abelian.
\Part Prove that if $G$ is an abelian group of order $pq$, where $p$ and $q$ are distinct primes, then $G$ is cyclic.
\Problem Let $R$ be a commutative ring with $1\ne 0$ whose only ideals are 0 and $R$. Show that $R$ is a field.
\Problem List all group homomorphisms from $\ZZ/40\ZZ$ to $\ZZ/60\ZZ$. Your answer must give a complete list, and indicate your notation clearly. You do not need to justify your answer.
\Problem Determine the greatest common divisor of 1761 and 1567.
\Problem Decide which of the following are ring homomorphisms from $M_2(\ZZ)$ to $\ZZ$:
\Part $\left(\begin{array}{cc}a & b \\c & d\end{array}\right)\mapsto a$
\Part $\left(\begin{array}{cc}a & b \\c & d\end{array}\right)\mapsto a+d$
\Part $\left(\begin{array}{cc}a & b \\c & d\end{array}\right)\mapsto ad-bc$

\Problem Decide which of the following are ideals of the ring $\ZZ\times\ZZ:$
\Part $\{ (a,a)\mid a\in \ZZ\}$
\Part $\{ (2a, 2b)\mid a,b\in\ZZ\}$
\Part $\{(2a,0)\mid a\in \ZZ\}$
\Part $\{(a,-a)\mid a\in \ZZ\}$
\Problem List all group homomorphisms from $\ZZ/40\ZZ$ to $\ZZ/60\ZZ$. Your answer must give a complete list, and indicate your notation clearly. You do not need to justify your answer.
\Problem Let $A$ and $B$ be groups. Prove that $A\times B\cong B\times A$.

\Problem Solve the simultaneous system of congruences \[x\equiv 3 \mod 16 \qquad x\equiv 9 \mod 25 \qquad y\equiv 42 \mod 49\]

\Problem Prove:
\Part Any group of order 35 is cyclic.
\Part Any group of order 147 is not simple.
\Problem Determine the greatest common divisor of 1761 and 1567.
\Problem Define $\phi:\CC^{\times}\to\RR^{\times}$ by $\phi(a+bi)=a^{2}+b^{2}$. Prove that $\phi$ is a homomorphism and find the image of $\phi$. Describe the kernel and fibers of $\phi$ geometrically (as subsets of the plane). 

\emph{Recall $S^{\times}$ is the set $S\setminus\{0\}$  under the usual multiplication operation.}
\Problem Give examples of each of the following or briefly explain why they can't exist:
\Part An integral domain that is not a field.
\Part A non-abelian simple group.
\Part An abelian simple group that is not cyclic.
\Part A non-abelian group with non-trivial center.
\Part A finitely generated abelian group that is not cyclic.
\Part An integral domain that is not a unique factorization domain.
\Part A principal ideal domain that is not a Euclidean domain.
\Part An infinite non-abelian group.
\Part A finite integral domain.
\Part A ring that is not an integral domain, but that is commutative.
\Problem Prove that if $A$ and $B$ are subsets of $G$ with $A\subseteq B$ then $C_{G}(B)$ is a subgroup of $C_{G}(A)$.
\Problem Let $r$ and $s$ be the usual generators for the dihedral group of order 8. List the elements of $D_{8}$ as $1, r, r^{2},r^{3},s,sr,sr^{2},sr^{3}$ and label these with the integers 1, 2, \ldots, 8 respectively. Exhibit the image of $D_{8}$ under the left regular representation of $D_{8}$ into $S_{8}$.
\Problem 
Prove that if $G$ is an abelian group of order $pq$, where $p$ and $q$ are distinct primes, then $G$ is cyclic.
\Problem Show that $f(x)=10x^{4}+6x^{3}+18x^{2}+6x+21 $ and $g(x)= x^3 - 5 x + 3$ are irreducible in $\QQ[x]$, and that $h(x)=x^{3}-5x+2$ is reducible.

\end{enumerate}


\bc\Large\textsc{Part 2: Group Theory}\ec

\noindent Complete 2 of the following problems. \emph{On the actual comprehensive exam, there will be four problems for you to choose from.} 
\begin{enumerate}
\Problem Suppose $G$ is a group with $H, K$ subgroups of $G$.  Prove that if $H \le N_G(K)$, then
$HK = \big \{ h k \mid h \in H, k \in K \big \}$ is  a subgroup of $G$. 

\Problem Suppose that a finite group $G$ is of order 105, $\vert G \vert = 3 \cdot 5 \cdot 7$, and that $G$ has normal subgroups of order $3$, $5$ and $7$.   Prove or disprove:  $G$ is cyclic.  

\Problem Let $P$ be a $p$-group, $\vert P \vert = p^a > 1$ for $p$ a prime, and let $A$ be a nonempty finite set.  Suppose that $P$ acts on $A$ and define
\emph{the set of fixed points} of this action:  
$$A_0 = \big \{ a \in A \mid g \cdot a = a \text{ for every } g \in P \big \}.$$


Prove that 
$$
\vert A \vert \equiv \vert A_0 \vert \Mod[p].
$$
\Problem Let $\eP (n)$ denote the Euler $\varphi$-function.  Prove that if $p$ is a prime and $n \in \ZZ^+$, then
$$
n \ldiv \eP (p^n - 1).
$$
(Hint:  Compute the order of $\bar p$ in the appropriate group first.)

\Problem Prove that if $G$ is a group of order $p^2$ for $p$ a prime, then $G$ is Abelian.

\Problem Suppose $G$ is a finite group of order $\vert G \vert = 14,553 = 3^3 \cdot 7^2 \cdot 11$ and that
$N$ is a normal subgroup of $G$ of order $\vert N \vert = 11$.  Prove that $N  \le Z(G)$.

\Problem Suppose $G$ is a group, $H \le G$, and $\Aut (H)$ the group of automorphisms
of $H$.  


\Part Using the First Isomorphism theorem, give a \tcm{\bf full} proof of the following statement.

The quotient group $N_G(H) / C_G(H) \cong A \le \Aut(H)$.

\Part Suppose now that $P$ is a Sylow $p$-subgroup of $S_p$ for a prime $p$.  Prove that
$$
N_{S_p} (P) / C_{S_p} ( P ) \cong \Aut(P).
$$


\Problem Let $G$ be a finite group of order 22.  Prove that $G$ is cyclic or isomorphic to the  dihedral group $D_{22}$.


\Problem Let $G$ be any group. Prove that the map from $G$ to itself defined by $g\mapsto g^{2}$ is a homomorphism if and only if $G$ is abelian. 
\Problem 
\Part Prove that if $\sigma:G\to G$ is the map $\sigma(x)=x^{-1}$ is a homomorphism, then $G$ is an abelian group.
\Part Let $G$ be a finite group which possesses an automorphism $\sigma$ such that $\sigma (g)=g$ if and only if $g=1$. If $\sigma^{2}$ is the identity map from $G$ to $G$, prove that $G$ is abelian. 

\noindent\emph{Hint:} Show that every element of $G$ can be written in the form $x^{-1}\sigma(x)$ and apply $\sigma$ to such an expression. 
\Problem Let $A$ be an abelian group and fix some $n\in\ZZ$. Prove that the following sets are subgroups of $A$:
\Part $H=\{a^n\mid a\in A\}$
\Part $K=\{a\in A\mid a^n=1\}$
\Problem Prove that the subgroup generated by any two distinct elements of order 2 in $S_{3}$ is all of $S_{3}$.

\Problem 
A group $H$ is \emph{finitely generated} if there is a finite set $A$ such that $H=\langle A\rangle$.
\Part Prove that every finite group is finitely generated.
\Part Prove that \ZZ is finitely generated.
\Part Prove that every finitely generated subgroup of the additive group \QQ is cyclic. [If $H$ is a finitely generated subgroup of \QQ, show that $H\le \langle \frac{1}{k}\rangle$, where $k$ is the product of all of the denominators which appear in a set of generators of $H$.]
\Part Prove that \QQ is not finitely generated.


\Problem Classify groups of order 4 by proving that if $|G|=4$ then $G\cong Z_{4}$ or $G\cong V_{4}$.

\Problem 
Let $\phi:G\to H$ be a homomorphism of groups with kernel $K$ and let $a,b\in \phi(G)$. Let $X\in G/K$ be the fiber above $a$ and let $Y$ be the fiber above $b$, i.e., $X=\phi^{-1}(a), Y=\phi^{-1}(b)$. Fix an element $u$ of $X$ (so $\phi(u)=a$). Prove that if $XY=Z$ in the quotient group $G/K$ and $w$ is any member of $Z$, then there is some $v\in Y$ such that $uv=w$. [Show $u^{-1}w\in Y$.]

\Problem 
Prove that in the quotient group $G/N$, $(gN)^{\alpha}=g^{\alpha}N$ for all $\alpha\in \ZZ$.


\Problem Let $\phi:\RR^\times\to\RR^\times$ be the map sending $x$ to the absolute value of $x$. Prove that $\phi$ is a homomorphism and find the image of $\phi$. Describe the kernel and fibers of $\phi$. 

\Problem Let $N$ be a finite subgroup of $G$ and suppose $G=\langle T\rangle$ and $N=\langle S\rangle$ for some subsets $T$ and $S$ of $G$. Prove that $N$ is normal in $G$ iff $tSt^{-1}\subseteq N$ for all $t\in T$.
\Problem
The \emph{join} of a non-empty collection of subgroups is the smallest subgroup that contains them all. Prove that the join of any non-empty collection of normal subgroups of a group is a normal subgroup. \emph{Hint: It may be helpful to show that $g\langle\cup_{i\in I} A_{i}\rangle g^{-1}=\langle\cup_{i\in I} gA_{i}g^{-1}\rangle$.}


\Problem Let $G$ be a finite group, let $H$ be a subgroup of $G$ and $N\normale G$. Prove that if $|H|$ and $|G:N|$ are relatively prime then $H\le N$.
\Problem Let $M$ and $N$ be normal subgroups of $G$ such that $G=MN$. Prove that $G/(M\cap N)\cong (G/M)\times(G/N)$.
\Problem Prove that if $H\normale G$ of prime index $p$, then for all $K\le G$ either (i) $K\le H$ or (ii) $G=HK$ and $|K:K\cap H|=p$.

\Problem If $G$ is a group of odd order, prove for any nonidentity $x\in G$ that $x$ and $x^{-1}$ are not conjugate in $G$.

\Problem Let $G$ be a group of order 203. Prove that if $H$ is a normal subgroup of order 7 in $G$ then $H\le Z(G)$. Deduce that $G$ is abelian in this case. 

\Problem Let $P$ be a Sylow $p$-subgroup of $H$ and let $H$ be a subgroup of $K$. If $P\trianglelefteq H$ and $H \trianglelefteq K$, prove that $P$ is normal in $K$.



\Problem If $A$ and $B$ are normal subgroups of $G$ such that $G/A$ and $G/B$ are both abelian, prove that $G/(A\cap B)$ is abelian. 

\Problem Prove that if $K$ is a normal subgroup of $G$, then $K'=\langle [x,y]\mid x,y\in K\rangle\trianglelefteq G$.
\Problem A group $G$ is a torsion group if every element has finite order. Prove: If $H\normal G$ and both $H$ and $G/H$ are torsion then $G$ is torsion.

\Problem Show that a group of order 150 has a normal subgroup of order 5 or 25.
\end{enumerate}

%%%%%%%%%%%%%%%%%%%%%%%%%%%%%%%%%%%%%%%%%%%%%%%%%%%%%%%%%%%%%%%
\bc\Large\textsc{Part 2: Ring and Field Theory}\ec
\noindent Complete 2 of the following problems. \emph{On the actual comprehensive exam, there will be four problems for you to choose from.} \begin{enumerate}
\Problem Prove that in a PID every nonzero element is a prime if, and only if,
it is irreducible.

\Problem Suppose $R$ is a commutative ring with 1 and for each $x \in R$, there is a positive integer $n > 1$ so that
$x^n = x$.  Prove that every nonzero prime ideal is maximal.

\Problem  Let $\FF_{7}$ denote the finite field with 7 elements.  
\Part Explicitly construct a finite field with $343 = 7^3$ elements.  Explain your work.

\Part The field you constructed in part (a) is a simple extension of $\FF_{7}$ so let
$\alpha$ be an element in some extension of $\FF_7$ such that
$\dsp \vert \FF_{7} (\alpha) \vert = 343$.
Find the inverse of the element $1 + \alpha \in \FF_{7} (\alpha)$.



\Problem Find, with brief justification, all ring homomorphisms from $\ZZ \to \ZZ/12\ZZ$.

\Problem Consider the ring of Gaussian integers $\ZZ[i]$. 

\Part Prove that if $\alpha = a + b i$ for $a, b \in \ZZ$ is a Gaussian integer with $N(\alpha) = p$ for $p$ a 
prime of $\ZZ$, then $\alpha$ is irreducible.

\Part List all the units of $\ZZ[i]$.

\Part Give an example of a prime number $p \in \ZZ$ such that $p$ is irreducible in $\ZZ[i]$.
Justify your answer by stating an appropriate result.




\Problem Let $F$ be a field and define the ring $F((x))$ of {\em formal Laurent series} with coefficients from $F$ by \[F((x))=\left\{ \sum_{n\ge N}^\infty a_n x^n\mid a_n\in F \text{ and } N\in \ZZ\right\}.\]

 Prove that $F((x))$ is a field. 
 
\Problem Let $R$ be a ring, $U,V$ ideals of $R$ such that $R/U$ and $R/V$ are commutative with $U\cap V=\{0\}$. Prove that $R$ is commutative.
\Problem Let $R$ be the ring of all continuous real valued functions on the closed interval $\left[0,1\right]$. Prove that the map $\phi:R\to\RR$ defined by $\phi(f)=\int_0^1f(t)dt$ is a homomorphism of additive groups but not a ring homomorphism.
\Problem Let $R$ and $S$ be nonzero rings with identity and denote their respective identities by $1_R$ and $1_S$. Let $\phi:R\to S$ be a nonzero homomorphism of rings.
\Part Prove that if $\phi(1_R)\ne 1_S$ then $\phi(1_R)$ is a zero divisor in $S$. Deduce that if $S$ is an integral domain then every ring homomorphism from $R$ to $S$ sends the identity of $R$ to the identity of $S$.
\Part Prove that if $\phi(1_R)=1_S$ then $\phi(u)$ is a unit in $S$ and that $\phi(u^{-1})=\phi(u)^{-1}$ for each unit $u$ of $R$.
\Problem Let $I$ and $J$ be ideals of $R$.
\Part Prove that $I+J$ is the smallest ideal of $R$ containing both $I$ and $J$.
\Part Prove that $IJ$ is an ideal contained in $I\cap J$.
\Part Give an example where $IJ\ne I\cap J$.
\Part Prove that if $R$ is commutative and if $I+J=R$ then $IJ=I\cap J$.



\Problem Assume $R$ is commutative ring with identity $1\ne 0$ and for each $a\in R$ there is an integer $n>1$ (depending on $a$) such that $a^n=a$. Prove that every prime ideal of $R$ is a maximal ideal. 

\Problem Let $R$ be a Euclidean domain. Let $m$ be the minimum integer in the set of norms of nonzero elements of $R$. Prove that every nonzero element of $R$ of norm $m$ is a unit. Deduce that a nonzero element of norm zero (it such an element exists) is a unit.


\Problem Let $R$ be a principal ideal domain.
\Part Prove that if $P$ is a prime ideal in $R$, the $P$ is maximal.
\Part Prove that if $M$ is a maximal ideal of $R$, then $R/M$ is a field.
\Part Prove that a quotient of a PID by a prime ideal is again a PID.

\Problem Let $I=(a)$ be a principal ideal in a commutative ring $R$ with 1. Suppose $P$ is a prime ideal such that $P\subsetneq I$. Prove that $P\subset \cap_{n\ge 1} I^n$.

\Problem Prove that $(x,y)$ and $(2,x,y)$ are prime ideals in $\ZZ(x,y)$, but only the latter ideal is a maximal ideal. 

\Problem Let $F$ be a finite field. Prove that $F[x]$ contains infinitely many primes.


\Problem Let $R=\ZZ+x\QQ[x]\subset \QQ[x]$ be the set of polynomials in $x$ with rational coefficients whose constant term is an integer.
\Part Prove that $R$ is an integral domain and its units are $\pm 1$.
\Part Show that the irreducibles in $R$ are $\pm p$ where $p$ is a prime in $\ZZ$ and the polynomials $f(x)$ that are irreducible in $\QQ[x]$ and have a constant term $\pm 1$. prove that these irreducibles are prime in $R$.
\Part Show that $x$ cannot be written as the product of irreducibles in $R$ (in particular, that $x$ is not irreducible) and conclude that $R$ is not a UFD.
\Part Show that $x$ is not a prime in $R$ and describe the quotient ring $R/(x)$.


\Problem Show that the polynomial $(x-1)(x-2)\cdots (x-n)-1$ is irreducible over $\ZZ$ for all $n\ge 1$. (Hint: If the polynomial factors consider the values of the factors at $x=1, 2, \ldots, n$.)
\Problem Prove that $K_{1}=\FF_{11}[x]/(x^{2}+1)$ and $K_{2}=\FF_{11}[y]/(y^{2}+2y+2)$ are both fields with 121 elements. Prove that the map which sends the element $p(\ol x)$ of $K_{1}$ to the element $p(\ol y +1)$ of $K_{2}$ (where $p$ any polynomial with coefficients in $\FF_{11}$) is well defined and gives a ring (hence field) isomorphism from $K_{1}$ to $K_{2}$.
\Problem Let $F$ be a field and let $f(x)$ be a polynomial of degree $n$ in $F[x]$. The polynomial $g(x)=x^{n}f(1/n)$ is called the {\em reverse} of $f(x)$. Describe the coefficients of $g$ in terms of the coefficients of $f$. If $f(0)\ne 0$ prove that $f$ is irreducible iff $g$ is irreducible. 
\Problem Prove that $x^{3}+12x^{2}+18x+6$ is irreducible over $\ZZ[i]$.




\end{enumerate}

\end{document}


 