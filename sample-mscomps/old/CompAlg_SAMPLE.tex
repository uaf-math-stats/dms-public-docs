\documentclass{report}
\usepackage{amsthm, amsmath, amssymb,verbatim,xspace, graphicx}
\usepackage{pdfsync, listings, color,pgf,tikz}
\usepackage[all,cmtip]{xy}

\newcommand\R{{\mathbb R}}
\newcommand\Q{{\mathbb Q}}
\newcommand\N{{\mathbb N}}
\newcommand\Z{{\mathbb Z}}
\newcommand\dsp{\displaystyle}

\newcommand\eP{\varphi}
\newcommand\ns{\trianglelefteq} 
\newcommand\ldiv{\, \bigg | \,}
\newcommand{\Mod[1]}{~( \operatorname{mod} \,#1)}
\newcommand{\Aut}{\operatorname{Aut}}
\newcommand\F{{\mathbb F}}

\newcommand{\tcm}{\textcolor{magenta}}

\setlength{\oddsidemargin}{-.25in} 
\setlength{\evensidemargin}{0in} 
\setlength{\textwidth}{7in} 
\setlength{\textheight}{9.5in} 

\setlength{\topmargin}{-.25in}
\setlength{\headheight}{0in}
\setlength{\headsep}{0in}

\usepackage{textcomp}

\newcommand{\tc}{\text{:}} % tc=tight colon, for use in math mode in Newick trees

\begin{document}

\noindent

\bigskip

\centerline{\sc \LARGE Sample Algebra Comprehensive Exam Problems}

\medskip

\centerline{Spring 2019}

\vskip .5cm 

This sample exam is a composite of questions asked on recent MATH 631 exams.  
As a result the length of this exam is much longer (by a factor of at least 2)
than a comprehensive exam, but the breadth
of topics covered and the difficulty level are representative of possible test questions.

\medskip

\noindent {\bf Part I.}  Short answer or easy computation.

\begin{enumerate}

\item Consider the cylic group $C_{4900} = \langle x \rangle$ of order $4900 = 2^2 \cdot 5^2 \cdot 7^2$. 

\begin{enumerate}

\item Give the number of generators of $C_{4900}$.

\item List explicitly the elements $x^a$, with $0 \le a \le 4899$, of order $10$.  

\tcm{\emph{Answer:}}  $\vert x^a \vert = 10$ if $a = \underline{\hskip 12cm }$.

\medskip

(If it helps, you can simply give the prime factorizations of $a$.  I am not interested in your ability to multiply integers.)

\end{enumerate}

\item Consider the cyclic groups $\Z/30\Z$ and $C_{18} = \langle x \rangle$ of orders $30$ and $18$ respectively, and
suppose that 
\begin{align*}
\varphi_a : \Z/30\Z &\to C_{18}\\
 1  &\mapsto x^a
\end{align*}
extends to a well-defined group homomorphism from $\Z/30\Z$ to $C_{18}$.

\begin{enumerate}

\item List the values of $a$ with $0 \le a \le 17$ for which this is true.  (I.e. The map defines a well-defined group
homomorphism.)

\item Give a brief explanation why such a well-defined group homomorphism can not be surjective.

\end{enumerate}

\item Consider the symmetric group $G = S_{7}$ and let $\sigma = (1 \ 2 \ 3 \ 6 \ 5 \ 4 \  7 )$ be a $7$-cycle.  

\begin{enumerate}

\item Express $\sigma$ as the product of (not necessarily disjoint) transpositions.

\item Compute the number of conjugates of $\sigma$ in $S_7$.

\item Let $\tau$ be the $7$-cycle $(3 \ 7 \ 1 \ 4 \ 5 \ 6 \ 2)$.  Give an element $\alpha$ that conjugates
$\sigma$ to $\tau$, i.e.  give $\alpha$ such that $\alpha \sigma {\alpha}^{-1} = \tau$.

\item Noting that $S_7$ acts on itself by conjugation, explicitly use the Orbit-Stabilizer theorem to find the 
size of the stabilizer of $\sigma$ under this action and the elements of the Stabilizer subgroup
of $S_7$.  

\tcm{The stabilizer of $\sigma$ in this context  is better known as  \underline{\hskip 3cm}. (Using appropriate notation in place of words 
here is fine.)}

\item Noting that $\sigma \in A_7$, what is the size of the conjugacy class of $\sigma$ in $A_7$?
Stated otherwise, how many conjugates in $A_7$ does $\sigma$ have?  Briefly, state a result that justifies
your answer.

\tcm{\emph{Answer:} The number of conjugates of $\sigma$ in $A_7$ is \underline{\hskip 2cm}} 

\medskip

\tcm{because ....}

\end{enumerate}

\item 

\begin{enumerate}

\item Suppose that $A$ is an Abelian group of order $200 =  2^3 \cdot 5^2$.  Give the isomorphism
classes for $A$ in a table below.  In the left hand column, give the elementary divisor decomposition
and in the right hand column, give the invariant factor decomposition.  {\bf Groups on the same row should
be isomorphic.}  You do not need to show your work.

\item  Give the number of non-isomorphic Abelian groups of order $400 = 2^4 \cdot 5^2$.

\end{enumerate}

\item Prove that there are no simple groups of order 56.

\item Give the definition of a nilpotent element in a ring $R$.  Then
prove that the set of nilpotent elements in $M_2(\Q)$ is {\bf not} an ideal.

\item  Suppose $G$ is a non-cyclic group of order $205 = 5\cdot 41$.  Give, with proof, the number
of elements of order $5$ in $G$.

\item 
Find {\bf ALL} solutions $x$ in the integers to the simultaneous congruences. 
\begin{align*}
x \equiv \ &7 \mod{11}\\
x \equiv \ &2 \mod{5}
\end{align*}

\item Draw the lattice diagram of prime ideals for the polynomial ring $\Q[x]$.  \emph{Note:}  There are infinitely many prime ideals so you 
will need a way to indicate them all.

\end{enumerate}

\vskip 1cm

\noindent {\bf Part II.}  Theory

\begin{enumerate}

\item Suppose $G$ is a group with $H, K$ subgroups of $G$.  Prove that if $H \le N_G(K)$, then
$HK = \big \{ h k \mid h \in H, k \in K \big \}$ is  a subgroup of $G$.

\item Suppose that a finite group $G$ is of order 105, $\vert G \vert = 3 \cdot 5 \cdot 7$, and that $G$ has normal subgroups
of order $3$, $5$ and $7$.   Prove or disprove:  $G$ is cyclic.

\item Let $P$ be a $p$-group, $\vert P \vert = p^a > 1$ for $p$ a prime, and let $A$ be a nonempty finite set.  Suppose that $P$ acts on $A$ and define
\emph{the set of fixed points} of this action:  
$$A_0 = \big \{ a \in A \mid g \cdot a = a \text{ for every } g \in P \big \}.$$


Prove that 
$$
\vert A \vert \equiv \vert A_0 \vert \Mod[p].
$$

\item Let $\eP (n)$ denote the Euler $\varphi$-function.  Prove that if $p$ is a prime and $n \in \Z^+$, then
$$
n \ldiv \eP (p^n - 1).
$$
(Hint:  Compute the order of $\bar p$ in the appropriate group first.)

\item Prove that if $G$ is a group of order $p^2$ for $p$ a prime, then $G$ is Abelian.

\item Suppose $G$ is a finite group of order $\vert G \vert = 14,553 = 3^3 \cdot 7^2 \cdot 11$ and that
$N$ is a normal subgroup of $G$ of order $\vert N \vert = 11$.  Prove that $N  \le Z(G)$.

\item Suppose $G$ is a group, $H \le G$, and $\Aut (H)$ the group of automorphisms
of $H$.  

\begin{enumerate}

\item Using the First Isomorphism theorem, give a \tcm{\bf full} proof of the following statement.

The quotient group $N_G(H) / C_G(H) \cong A \le \Aut(H)$.

\item Suppose now that $P$ is a Sylow $p$-subgroup of $S_p$ for a prime $p$.  Prove that
$$
N_{S_p} (P) / C_{S_p} ( P ) \cong \Aut(P).
$$

\end{enumerate}

\item Let $G$ be a finite group of order 22.  Prove that $G$ is cyclic or isomorphic to the 
dihedral group $D_{22}$.

\item In a PID every nonzero element is a prime if, and only if,
it is irreducible.

\item Suppose $R$ is a commutative ring with 1 and for each $x \in R$, there is a positive integer $n > 1$ so that
$x^n = x$.  Prove that every nonzero prime ideal is maximal.

\item  Let $\F_{7}$ denote the finite field with 7 elements.  

\begin{enumerate}

\item Explicitly construct a finite field with $343 = 7^3$ elements.  Explain your work.

\item The field you constructed in part (a) is a simple extension of $\F_{7}$ so let
$\alpha$ be an element in some extension of $\F_7$ such that
$\dsp \vert \F_{7} (\alpha) \vert = 343$.
Find the inverse of the element $1 + \alpha \in \F_{7} (\alpha)$.

\end{enumerate}

\item Find, with brief justification, all ring homomorphisms from $\Z \to \Z/12Z$.

\item Consider the ring of Gaussian integers $\Z[i]$.  

\begin{enumerate}

\item Prove that if $\alpha = a + b i$ for $a, b \in \Z$ is a Gaussian integer with $N(\alpha) = p$ for $p$ a 
prime of $\Z$, then $\alpha$ is irreducible.

\item List all the units of $\Z[i]$.

\item Give an example of a prime number $p \in \Z$ such that $p$ is irreducible in $\Z[i]$.
Justify your answer by stating an appropriate result.

\end{enumerate}

\item  Let $D$ be a square-free integer, and consider the quadratic number field $\Q (\sqrt{D})$ and its subring of integers
$\mathcal O$.  Let $N: \Q(\sqrt{D}) \to \Z$ denote the field norm map which is multiplicative.  The restriction of $N$ to
the ring of integers $\mathcal O$ will also denoted by $N$.

\begin{enumerate}

\item Prove that an element $\alpha \in \mathcal O$ is a unit if, and only if, $N(\alpha) = \pm1$.

\item  When $D = -3$, the ring of integers is $\dsp \mathcal O = \Z + \Z \bigg(\frac{1 + \sqrt{-3}}{2}\bigg)$.
Find a unit in $\mathcal O \smallsetminus \Z$.  

\item  Let $D=-5$.  Give, with proof, an example of an element $x = a + b \sqrt{-5}$ for $a, b \in Z$
such that $x$ is irreducible, but $x$ is not prime in $\Z [ \sqrt{-5}]$.  
\end{enumerate}

\end{enumerate}

\end{document}
